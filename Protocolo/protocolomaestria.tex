%\documentclass[11pt,epsf,times,twocolumn]{article}
\documentclass[11pt,epsf,times]{article}
\usepackage{epsf,latexsym}
\usepackage[spanish]{babel}
\usepackage[latin1]{inputenc}
%\usepackage{graphicx,moreverb}
\hoffset=-19pt
\voffset=-36pt
\textheight=244mm
%\textheight=680p
\textwidth=505pt
\marginparsep=30pt
\columnsep=9.9mm
%\columnsep=20pt
\def\figurename{Figura}
\pagestyle{empty}

\pagestyle{plain}
\textwidth 6.5in
\textheight 8.75in
\oddsidemargin 0in
\evensidemargin 0in
\topmargin -0.5in
\newcommand{\pp}[1]{$\langle$#1$\rangle$}
%-----------------------

\title{ Centro de Investigaci\'{o}n y Estudios Avanzados del IPN\\
  Unidad Tamaulipas\\
  \textbf{Protocolo de tesis}
}


\author{
Título: Estrategias para la exploración coordinada multi-VANT \\
Candidato: Luis Alberto Ballado Aradias \\
Asesor: Dr. José Gabriel Ramirez Torres \\
Co-Asesor: Dr. Eduardo Rodriguez Tello
}

\date{\today}

\usepackage{amssymb}
\usepackage{pgfgantt}
\usepackage[spanish]{babel}
\usepackage[latin1]{inputenc}
\selectlanguage{spanish}

\addtolength{\textheight}{90pt}

\newcommand{\I}{\mathbb{I}}
\newcommand{\K}{\mathbb{K}}
\newcommand{\N}{\mathbb{N}}
\newcommand{\Q}{\mathbb{Q}}
\newcommand{\R}{\mathbb{R}}
\newcommand{\Z}{\mathbb{Z}}

\begin{document}
\maketitle

\begin{abstract}

  %La importancia de la robótica de servicios y los Véhiculos Aéreos No Tripulados (VANT) en la actualidad es innegable. Estas tecnologías están revolucionando la forma en que interactuamos con el mundo y ofrecen un amplio abanico de aplicaciones en diversos sectores. Desde la entrega de paquetes y la asistencia en emergencias hasta la exploración espacial, la robótica de servicios, ha demostrado ser especialmente valiosa en entornos donde los seres humanos pueden enfrentar riesgos o dificultades. Estos robots pueden trabajar de manera autónoma o en colaboración.\\
  %Los Véhiculos Aéreos No Tripulados, tienen la capacidad de volar y acceder a lugares de manera rápida y eficiente convirtiéndolos en herramientas extremadamente versátiles. Por ejemplo, los VANT(s) están siendo utilizados por empresas de comercio electrónico para entregar productos a los clientes de manera más ágil. Además, en la agricultura se utilizan para monitorear cultivos, identificar problemas y aplicar pesticidas o fertilizantes de manera precisa, lo que aumenta la eficiencia y logra reducir costos. En el ámbito de la seguridad, pueden utilizarse para la vigilancia de áreas de difícil acceso o en situaciones de emergencia, proporcionando información valiosa en tiempo real a los equipos de rescate.\\

  %A pesar de los numerosos beneficios de la robótica de servicios y los Véhiculos Aéreos No Tripulados, existen desafíos y problemáticas asociadas a su implementación.\\
  %Una de ellas es la \textbf{exploración}, la capacidad de exploración amplía nuestro conocimiento y nos ayuda a comprender mejor el mundo que nos rodea. \\
  %La \textbf{coordinación} y el \textbf{trabajo en equipo} de múltiples-VANTs representa un desafío emocionante. La \textbf{colaboración} de varios VANT(s) puede ser utilizada en misiones de búsqueda y rescate, en donde pueden cubrir áreas más extensas y realizar tareas más complejas de manera simultánea. La coordinación entre los drones puede optimizar la eficiencia de las operaciones y aumentar las posibilidades de éxito.\\

  %La implementación de sistemas de múltiples VANT(s) también plantea desafíos técnicos y logísticos. La \textbf{comunicación} entre ellos, la asignación de tareas y la planificación de rutas con comportamientos reactivos para la detección y evición de obstáculos son aspectos críticos que deben abordarse para garantizar un funcionamiento fluido y seguro.\\

  %El objetivo de este trabajo es la propuesta de una arquitectura de software resiliente a fallas, capaz de explorar ambientes desconocidos y cambiantes para la coordinación de Vehículos Aéreos No Tripulados. El proyecto de investigación demostrará que es posible diseñar algoritmos inteligentes de poca memoria capaces de resolver tareas en colaboración multi-VANT.
  %\medskip\\
  
  %\noindent \textsf{Palabras claves:} multi-VANT, coordinación multi-agente, Exploración 3D, 3D Path finding
  
\end{abstract}

\newpage
\section{Datos Generales}

\subsection{T\'{\i}tulo de proyecto}
Estrategias para la exploraci\'{o}n coordinada multi-VANT

\subsection{Datos del alumno}
\begin{tabular}{ll} 
Nombre:  &          Luis Alberto Ballado Aradias \\
Direcci\'{o}n:   & Juan José de La Garza \#909\\
                 & Colonia: Guadalupe Mainero C.P. 87130\\
Tel\'{e}fono (casa):    & 81 20706661 \\
Tel\'{e}fono (lugar de trabajo):    & (834) 107 0220 + Ext  \\
Direcci\'{o}n electr\'{o}nica: & luis.ballado@cinvestav.mx \\
URL: & https://luis.madlab.mx
\end{tabular}
\subsection{Instituci\'{o}n}
\begin{tabular}{ll} 
Nombre:  &          CINVESTAV-IPN \\
Departamento:    &  Unidad Tamaulipas\\
Direcci\'{o}n:   &  Km 5.5 carretera Cd. Victoria - Soto la Marina.\\
                 &  Parque Cient\'{i}fico y Tecnol\'{o}gico TECNOTAM,\\
                 &  Ciudad Victoria, Tamaulipas, C.P. 87130\\
Tel\'{e}fono:    & (+52) (834) 107 0220\\
\end{tabular}
\subsection{Beca de tesis}
\begin{tabular}{ll} 
Instituci\'{o}n otorgante:  &  CONAHCYT  \\
Tipo de beca:      & Maestr\'ia Nacional\\
Vigencia:    &   Septiembre 2022 - Agosto 2024
\end{tabular}

\subsection{Datos del asesor}
\begin{tabular}{ll} 
Nombre:  &   Dr. José Gabriel Ramírez Torres \\
Direcci\'{o}n:   &   Km. 5.5 carretera Cd. Victoria - Soto la Marina\\
                 &  Parque Cient\'{i}fico y Tecnol\'{o}gico TECNOTAM\\
                 &  Ciudad Victoria, Tamaulipas, C.P. 87130\\
Tel\'{e}fono (oficina):    &  (+52) (834) 107 0220 Ext. 1014 \\ 
Instituci\'{o}n:    &  CINVESTAV-IPN \\ 
Departamento adscripci\'{o}n: &  Unidad Tamaulipas\\
Grado acad\'{e}mico: & Doctorado \\\\
Nombre:  &   Dr. Eduardo Arturo Rodríguez Tello \\
Direcci\'{o}n:   &   Km. 5.5 carretera Cd. Victoria - Soto la Marina\\
                 &  Parque Cient\'{i}fico y Tecnol\'{o}gico TECNOTAM\\
                 &  Ciudad Victoria, Tamaulipas, C.P. 87130\\
Tel\'{e}fono (oficina):    &  (+52) (834) 107 0220 Ext. 1100\\ 
Instituci\'{o}n:    &  CINVESTAV-IPN \\ 
Departamento adscripci\'{o}n: &  Unidad Tamaulipas\\
Grado acad\'{e}mico: & Doctorado 
\end{tabular} 

\newpage
\section{Descripci\'{o}n del proyecto}

El proyecto de estrategias para la exploración coordinada multi-VANT se centra en el desarrollo de técnicas y algoritmos que permiten la coordinación eficiente de múltiples drones para llevar a cabo tareas de exploración en entornos desconocidos y cambiantes. \\
El objetivo principal es aprovechar las ventajas de tener multiples-VANT(s) trabajando en conjunto para mejorar la eficiencia y la cobertura de la exploración.\\

%Generación de una arquitectura descentralizada capaz de coordinar la exploración de multiples Vehículos Aéreos No Tripulados (VANTs) en ambientes desconocidos y cambiantes.

\subsection{Antecedentes y motivaci\'{o}n para el proyecto}

Millones de Véhiculos Aéreos No Tripulados, también conocidos como VANT(s) o drones, han presentado una adopción masiva en diferentes aplicaciones, desde usos civiles (búsqueda y rescate, monitoreo industrial, vigilancia), hasta aplicaciones militares [1]. La popularidad de los VANT(s) es atribuida a su movilidad.\\

Aplicaciones que hacen uso de los VANT(s) con sensores abordo son sorprendentes, ya que utilizan las ventajas de sensores eficientes [1] (sensores termicos, cámaras fotográficas,..etc.). Los VANT(s) han demostrado ser útiles en aplicaciones industriales [2] petroleras, parques eólicos, etc. Los VANT(s) pueden ayudar en la recolección de información.\\

El interes en la investigación e inovación de soluciones con Véhiculos Aéreos No Tripulados ha crecido exponencialmente recientemente [7,8,9,10].\\

En recientes años, dotar a los VANT de inteligencia para explotar la información recolectada de sensores a bordo, ha sido y es un área estudiada en robótica móvil áerea (Construcción de Mapas)[3]. Buscando probar diferentes teorías de control, convirtiendo los problemas típicos de control 2D (péndulo inverso fijo) a un ambiente 3D, teniendo así más variables a controlar buscando mantener el equilibrio del péndulo y al mismo tiempo lograr el movimiento y las maniobras deseadas del dron en el espacio tridimensional[4].\\

El despliegue rápido de robots en situaciones de riesgo, búsqueda y rescate ha sido un área ampliamente estudiada en la robótica móvil. Donde se han aplicado teorias de grafos para la optención de la mejor ruta. Los comportamientos reactivos son primordiales si pensamos en un agente autonomo. Esa percepción que podemos tener los seres humanos para reaccionar a ciertos retos. Buscar la manera de crear una arquitectura tolerante a fallas y capaz de coordinar multiples vehiculos aereos no trupulados a medida que incrementa o disminuye la oferta de drones disponibles


\section{Planteamiento del problema}

Para poder desplazarse en un ambiente desconocido lo primero es explorar el area y a medida que se obtiene información del espacio se puden calcular las rutas mas cortas o ciertas caracteristicas que nos ayuden a conocer donde estoy dentro de un ambiente desconocido. Estudios en robotica movil por mas de 25 años han demostrado que la teoria de grafos ha aydado mucho al area ya que se representaba como grafos planos y ahora al tener el agente robot en el aire, el problema se vuelve 3D.\\

Dada un área de interés $A$ que se desea explorar,
\begin{itemize}
\item Un conjunto de Véhiculos Aéreos No Tripulados (VANT) denotados como $V = V_{1},V_{2},V_{3},...,V_{n}$, donde $n$ es el número total de VANT's disponibles
\item Un conjunto de tareas de exploración denotados como $T = T_{1}, T_{2}, T_{3}, T_{m}$, donde $m$ es el número total de tareas a realizar.
\end{itemize}

restricciones y requisitos específicos del problema, como límites de tiempo, obstáculos a evitar, etc.

Para cada tarea de exploración $T_{m}$, se definen las siguientes variables:

\begin{itemize}
\item Posición inicial: $p_{i}(x,y,z)$, representa la posición inicial del VANT o los múltiples-VANTs asignados a la tarea $T_{m}$
\item Trayectoria: $\alpha_{i}$, describe la trayectoria seguida por el/los VANT(s) asignado(s) a la tarea $T_{m}$ en función del tiempo $t$
\item Información recolectada: $C_{i}$, representa la información recolectada por el/los VANT(s) asignado(s) durante la exploración
\end{itemize}

La función objetivo variará según los objetivos específicos del problema.
\begin{itemize}
\item Maximizar la cobertura del área de interés $A$
\item Minimizar el tiempo total requerido para cubrir el área de interés $A$
\item Maximizar la cantidad de información recolectada
\end{itemize}

\newpage
\section{Objetivos generales y espec\'{\i}ficos del proyecto}

\textbf{General} \\

Creación de una arquitectura tolerante a fallas para la coordinación de multiples VANT's aplicados a la tarea de búsqueda y rescate.

\bigskip
\noindent
\textbf{Particulares} \\
\begin{itemize}
\item Evitar colisiones: El objetivo primordial es garantizar que los drones eviten colisiones entre sí y con otros objetos en su entorno. La coordinación adecuada asegura que los drones mantengan una distancia segura y sigan rutas que minimicen el riesgo de colisión.
\item Eficiencia y rendimiento: La coordinación de múltiples drones busca optimizar la eficiencia y el rendimiento del sistema en su conjunto. Esto implica asignar tareas de manera óptima entre los drones, minimizar los tiempos de espera y los tiempos de respuesta, y maximizar la capacidad de realizar tareas en paralelo.
\item Cumplimiento de objetivos: Los drones pueden tener objetivos específicos a cumplir, como la recolección de datos, la entrega de paquetes o la vigilancia de áreas. La coordinación tiene como objetivo garantizar que cada dron contribuya de manera efectiva al logro de los objetivos generales, sin redundancia ni superposición de tareas.
\item Distribucion equitativa de tareas: Si los drones tienen capacidades o limitaciones diferentes, como la duración de la batería o la capacidad de carga, se busca una distribución equitativa de la carga de trabajo entre los drones. Esto asegura que todos los drones contribuyan de manera equilibrada y evita la sobrecarga de algunos drones mientras otros están subutilizados.
\item Comunicación y sincronicación: La coordinación requiere una comunicación efectiva entre los drones para intercambiar información y sincronizar sus acciones. El objetivo es establecer una comunicación confiable y eficiente que permita la transmisión de datos relevantes y las instrucciones necesarias para la coordinación.
\item Adaptabilidad y flexibilidad: Los objetivos de la coordinación pueden cambiar en función de las circunstancias y las necesidades. La coordinación de múltiples drones debe ser adaptable y flexible para ajustarse a cambios en el entorno, nuevos objetivos o la incorporación o salida de drones del sistema.
\end{itemize}

\section{Metodolog\'{\i}a}

La metodología propuesta para esta tesis se divide en tres etapas, iniciando en septiembre del 2023. A continuación se detallan cada una de las actividades que se plantean realizar en cada una.

\begin{enumerate}
\item Análisis y diseño de la solución propuesta
\item Implementación y validación
\item Evaluación, resultados y conclusiones
\end{enumerate}

\newpage
\section{Cronograma de actividades (plan de trabajo)}

\begin{ganttchart}[vgrid={draw=none, dotted}]{1}{12}
\gantttitlelist{1,...,12}{1} \\
\ganttbar{}{1}{4} \\
\ganttbar{}{5}{11}
\end{ganttchart}

\iffalse
\definecolor{barblue}{RGB}{153,204,254}
\definecolor{groupblue}{RGB}{51,102,254}
\definecolor{linkred}{RGB}{165,0,33}
\renewcommand\sfdefault{phv}
\renewcommand\mddefault{mc}
\renewcommand\bfdefault{bc}
\setganttlinklabel{s-s}{START-TO-START}
\setganttlinklabel{f-s}{FINISH-TO-START}
\setganttlinklabel{f-f}{FINISH-TO-FINISH}
\sffamily
\begin{ganttchart}[
    canvas/.append style={fill=none, draw=black!5, line width=.75pt},
    hgrid style/.style={draw=black!5, line width=.75pt},
    vgrid={*1{draw=black!5, line width=.75pt}},
    today=7,
    today rule/.style={
      draw=black!64,
      dash pattern=on 3.5pt off 4.5pt,
      line width=1.5pt
    },
    today label font=\small\bfseries,
    title/.style={draw=none, fill=none},
    title label font=\bfseries\footnotesize,
    title label node/.append style={below=7pt},
    include title in canvas=false,
    bar label font=\mdseries\small\color{black!70},
    bar label node/.append style={left=2cm},
    bar/.append style={draw=none, fill=black!63},
    bar incomplete/.append style={fill=barblue},
    bar progress label font=\mdseries\footnotesize\color{black!70},
    group incomplete/.append style={fill=groupblue},
    group left shift=0,
    group right shift=0,
    group height=.5,
    group peaks tip position=0,
    group label node/.append style={left=.6cm},
    group progress label font=\bfseries\small,
    link/.style={-latex, line width=1.5pt, linkred},
    link label font=\scriptsize\bfseries,
    link label node/.append style={below left=-2pt and 0pt}
  ]{1}{13}
  \gantttitle[
    title label node/.append style={below left=7pt and -3pt}
  ]{CUATRIMESTRE:\quad1}{1}
  \gantttitlelist{2,...,13}{1} \\
  \ganttgroup[progress=57]{WBS 1 Summary Element 1}{1}{10} \\
  \ganttbar[
    progress=75,
    name=WBS1A
  ]{\textbf{WBS 1.1} Activity A}{1}{8} \\
  \ganttbar[
    progress=67,
    name=WBS1B
  ]{\textbf{WBS 1.2} Activity B}{1}{3} \\
  \ganttbar[
    progress=50,
    name=WBS1C
  ]{\textbf{WBS 1.3} Activity C}{4}{10} \\
  \ganttbar[
    progress=0,
    name=WBS1D
  ]{\textbf{WBS 1.4} Activity D}{4}{10} \\[grid]
  \ganttgroup[progress=0]{WBS 2 Summary Element 2}{4}{10} \\
  \ganttbar[progress=0]{\textbf{WBS 2.1} Activity E}{4}{5} \\
  \ganttbar[progress=0]{\textbf{WBS 2.2} Activity F}{6}{8} \\
  \ganttbar[progress=0]{\textbf{WBS 2.3} Activity G}{9}{10}
  \ganttlink[link type=s-s]{WBS1A}{WBS1B}
  \ganttlink[link type=f-s]{WBS1B}{WBS1C}
  \ganttlink[
    link type=f-f,
    link label node/.append style=left
  ]{WBS1C}{WBS1D}
  \end{ganttchart}
\fi

\section{Estado del arte}

\begin{itemize}
\item Planificador de trayectorias
\item Arquitectura de robots coordinados
\item Percepcion de información 3D a partir de sistemas de vision por computadora
\end{itemize}
\section{Contribuciones o resultados esperados}

Se espera entregar:

\begin{enumerate}
\item Códigos a disposición de la comunidad 
\item Modelo dinámico VANT 3D
\item Simulación 3D con el software libre Gazebo
\item Tesis impresa.
\end{enumerate}

\section{Referencias}
        [1] H. Shakhatreh et al., "Unmanned Aerial Vehicles: A Survey on Civil Applications and Key Research Challenges" arXiv:1805.00881, 2018\\
        [2] P. Daponte, L. De Vito, G. Mazzilli, F. Picariello, S. Rapuano, and M.Riccio, "Metrology for drone and drone for metrology: Measurement systems on small civilian drones," in Metrology for Aerospace (MetroAeroSpace), 2015 IEEE, 2015, pp. 306-311: IEEE.\\
        [3] A. Shukla and H. Karki, "Application of robotics in onshore oil and gas industry A review Part I," Robotics and Autonomous Systems, vol. 75, pp. 490-507, 2016\\
        [4] M. Hehn and R. D'Andrea, "A flying inverted pendulum," 2011 IEEE International Conference on Robotics and Automation, Shanghai, China, 2011, pp. 763-770, doi: 10.1109/ICRA.2011.5980244.\\
        [5]\\
        [6]\\
        [7] \\
        [8] L. Gupta, R. Jain, and G. Vaszkun, "Survey of important issues in UAV communication networks," IEEE Communications Surveys & Tutorials, vol. 18, no. 2, pp. 1123-1152, 2016.\\
        [9] J. Senthilnath, M. Kandukuri, A. Dokania, and K. Ramesh, "Application of UAV imaging platform for vegetation analysis based on spectral-spatial methods," Computers and Electronics in Agriculture, vol. 140, pp. 8-24, 2017.\\
        [10]\\
        
\newpage
\begin{center}
\begin{tabular}{c@{\hspace{5em}}c}
{\Large{Fecha de inicio}} & {\Large{Fecha de terminaci\'on}} \\
% Poner fechas respectivas
&\\
Septiembre de 2023 & Agosto de 2024
\end{tabular} \vspace{2.5cm} \\
Firma del alumno: \underline{\hspace{5cm}} \vspace{2cm}\\ \ \\
{\Large{Comit\'e de aprobaci\'on del tema de tesis}} \vspace{2cm} \\
\begin{tabular}{p{7cm}p{5cm}}
Dr. José Gabriel Ramírez Torres & \underline{\hspace{5cm}} \vspace{1cm} \\
Dr. Eduardo Arturo Rodríguez Tello   & \underline{\hspace{5cm}} \vspace{1cm} \\
%Dr. Francisco Rodr\'iguez Henr\'iquez & \underline{\hspace{5cm}} %\vspace{1cm} \\
\end{tabular}
\end{center}
%\newpage
%\bibliographystyle{plain}
%\bibliography{c:/RodRuiz/bib}
%\bibliography{book,jour,kocc,proc,trep}

\end{document}






