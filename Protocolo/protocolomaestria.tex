%\documentclass[11pt,epsf,times,twocolumn]{article}
\documentclass[11pt,epsf,times]{article}
\usepackage{epsf,latexsym}
\usepackage[spanish]{babel}
\usepackage[latin1]{inputenc}
%\usepackage{graphicx,moreverb}
\hoffset=-19pt
\voffset=-36pt
\textheight=244mm
%\textheight=680p
\textwidth=505pt
\marginparsep=30pt
\columnsep=9.9mm
%\columnsep=20pt
\def\figurename{Figura}
\pagestyle{empty}

\pagestyle{plain}
\textwidth 6.5in
\textheight 8.75in
\oddsidemargin 0in
\evensidemargin 0in
\topmargin -0.5in
\newcommand{\pp}[1]{$\langle$#1$\rangle$}
%-----------------------

\title{ Centro de Investigaci\'{o}n y Estudios Avanzados del IPN\\
  Unidad Tamaulipas\\
  \textbf{Protocolo de tesis}
}


\author{
T�tulo: Estrategias para la exploraci�n coordinada multi-VANT \\
Candidato: Luis Alberto Ballado Aradias \\
Asesor: Dr. Jos� Gabriel Ramirez Torres \\
Co-Asesor: Dr. Eduardo Rodriguez Tello
}

\date{\today}

\usepackage{amssymb}

\addtolength{\textheight}{90pt}

\newcommand{\I}{\mathbb{I}}
\newcommand{\K}{\mathbb{K}}
\newcommand{\N}{\mathbb{N}}
\newcommand{\Q}{\mathbb{Q}}
\newcommand{\R}{\mathbb{R}}
\newcommand{\Z}{\mathbb{Z}}

\begin{document}
\maketitle

\begin{abstract}

  La exploraci�n es un problema fundamental de la rob�tica m�vil aut�noma que consiste en utilizar un robot para obtener informaci�n de un entorno desconocido, a trav�s de sus sensores, con el objetivo de generar un mapa que lo represente. Por motivos de eficiencia y robustez la exploraci�n suele llevarse a cabo con m�s de un robot, en este caso el problema se conoce como el de exploraci�n multi-robot.\\
  
  Uno de los principales aspectos de la exploraci�n es determinar a que lugares los robots deben moverse para obtener informaci�n del entorno. Para esto se suele primero identificar dichos lugares, conocidos como objetivos de exploraci�n, para luego asignar los objetivos identificados a los robots. Cuando se utiliza m�s de un robot es deseable que la asignaci�n de objetivos se lleve a cabo siguiendo una estrategia de coordinaci�n para evitar que los robots exploren los mismos lugares o que se obstaculicen entre s�.\\
  
Los entornos estructurados como edificios, hogares y otras construcciones humanas son entornos que pueden dividirse en segmentos, como habitaciones y corredores. En este tipo de entornos una posible estrategia de coordinaci�n es la de llevar a cabo la exploraci�n maximizando la distribuci�n de los robots sobre los segmentos.
%\pp{Poner resumen}
\medskip \\
\noindent \textsf{Palabras claves:} \pp{Poner palabras claves}

\end{abstract}

\section{Datos Generales}

\subsection{T\'{\i}tulo de proyecto}
Estrategias para la exploraci�n coordinada multi-VANT
\subsection{Datos del alumno}
\begin{tabular}{ll} 
Nombre:  &          Luis Alberto Ballado Aradias \\
Direcci\'{o}n:   & \pp{Una l�nea}\\
                 & \pp{otra l�nea} \\
Tel\'{e}fono (casa):    & 833 2126651 \\
Tel\'{e}fono (lugar de trabajo):    & \pp{tel} \\
Direcci\'{o}n electr\'{o}nica: & luis.ballado@cinvestav.mx \\
URL: & luis.madlab.mx
\end{tabular}
\subsection{Instituci\'{o}n}
\begin{tabular}{ll} 
Nombre:  &          CINVESTAV-IPN \\
Departamento:    &  Unidad Tamaulipas\\
Direcci\'{o}n:   &  Km 5.5 carretera Cd. Victoria - Soto la Marina.\\
                 &  Parque Cient�fico y Tecnol�gico TECNOTAM,\\
                 &  Ciudad Victoria, Tamaulipas, C.P. 87130\\
Tel\'{e}fono:    & (+52) (834) 107 0220\\
\end{tabular}
\subsection{Beca de tesis}
\begin{tabular}{ll} 
Instituci\'{o}n otorgante:  &  CONACYT  \\
Tipo de beca:      & Maestr\'ia Nacional\\
Vigencia:    &   Septiembre 2022 - Agosto 2024
\end{tabular}

\subsection{Datos del asesor}
\begin{tabular}{ll} 
Nombre:  &   Dr. Jos� Gabriel Ramirez Torres \\
Direcci\'{o}n:   &   Km. 5.5 carretera Cd. Victoria - Soto la Marina\\
                 &  Parque Cient�fico y Tecnol�gico TECNOTAM\\
                 &  Ciudad Victoria, Tamaulipas, C.P. 87130\\
Tel\'{e}fono (oficina):    &  (+52) (834) 107 0220 Ext.  \\ 
Instituci\'{o}n:    &  CINVESTAV-IPN \\ 
Departamento adscripci\'{o}n: &  Unidad Tamaulipas\\
Grado acad\'{e}mico: & Doctorado \\\\
Nombre:  &   Dr. Eduardo Rodriguez Tello \\
Direcci\'{o}n:   &   Km. 5.5 carretera Cd. Victoria - Soto la Marina\\
                 &  Parque Cient�fico y Tecnol�gico TECNOTAM\\
                 &  Ciudad Victoria, Tamaulipas, C.P. 87130\\
Tel\'{e}fono (oficina):    &  (+52) (834) 107 0220 Ext.  \\ 
Instituci\'{o}n:    &  CINVESTAV-IPN \\ 
Departamento adscripci\'{o}n: &  Unidad Tamaulipas\\
Grado acad\'{e}mico: & Doctorado 
\end{tabular} 

\section{Descripci\'{o}n del proyecto}

\subsection{Antecedentes y motivaci\'{o}n para el proyecto}


\section{Planteamiento del problema}


\section{Objetivos generales y espec\'{\i}ficos del proyecto}

\textbf{General} \\

Se trata de \pp{lo que se trata}

\bigskip
\noindent
\textbf{Particulares} \\
\begin{enumerate}
\item \pp{particular uno}
\item \pp{particular dos}
\item \pp{particular cu�l sea necesario}
\end{enumerate}

\section{Metodolog\'{\i}a}

\begin{enumerate}
\item 
\item 
\item 
\item 
\end{enumerate}
 
\section{Cronograma de actividades (plan de trabajo)}
\pp{calendarizar por cuatrimestre}

\section{Estado del arte}

\section{Contribuciones o resultados esperados}
\pp{introducir esta parte}

Se espera entregar:
\begin{enumerate}
\item 
\item 
\item 
\item Tesis impresa.
\end{enumerate}

\section{Referencias}
\newpage
\begin{center}
\begin{tabular}{c@{\hspace{5em}}c}
{\Large{Fecha de inicio}} & {\Large{Fecha de terminaci\'on}} \\
% Poner fechas respectivas
&\\
Septiembre de 2002 & Agosto de 2003
\end{tabular} \vspace{2.5cm} \\
Firma del alumno: \underline{\hspace{5cm}} \vspace{2cm}\\ \ \\
{\Large{Comit\'e de aprobaci\'on del tema de tesis}} \vspace{2cm} \\
\begin{tabular}{p{7cm}p{5cm}}
Dr. Guillermo Morales Luna & \underline{\hspace{5cm}} \vspace{1cm} \\
Dr. Oscar Olmedo Aguirre   & \underline{\hspace{5cm}} \vspace{1cm} \\
Dr. Francisco Rodr\'iguez Henr\'iquez & \underline{\hspace{5cm}} %\vspace{1cm} \\
\end{tabular}
\end{center}
%\newpage
%\bibliographystyle{plain}
%\bibliography{c:/RodRuiz/bib}
%\bibliography{book,jour,kocc,proc,trep}

\end{document}






