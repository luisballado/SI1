%\documentclass[11pt,epsf,times,twocolumn]{article}
\documentclass[11pt,epsf,times]{article}
\usepackage{epsf,latexsym}
\usepackage[spanish]{babel}
\usepackage[latin1]{inputenc}
%\usepackage{graphicx,moreverb}
\hoffset=-19pt
\voffset=-36pt
\textheight=244mm
%\textheight=680p
\textwidth=505pt
\marginparsep=30pt
\columnsep=9.9mm
%\columnsep=20pt
\def\figurename{Figura}
\pagestyle{empty}

\pagestyle{plain}
\textwidth 6.5in
\textheight 8.75in
\oddsidemargin 0in
\evensidemargin 0in
\topmargin -0.5in
\newcommand{\pp}[1]{$\langle$#1$\rangle$}
%-----------------------

\title{ Centro de Investigaci\'{o}n y Estudios Avanzados del IPN\\
  Unidad Tamaulipas\\
  \textbf{Protocolo de tesis}
}

\author{
T\'{i}tulo: Estrategias para la exploraci\'{o}n coordinada multi-VANT \\
Candidato: Luis Alberto Ballado Aradias \\
Asesor: Dr. Jos\'{e} Gabriel Ram\'{i}rez Torres \\
Co-Asesor: Dr. Eduardo Arturo Rodr\'{i}guez Tello
}


\date{\today}

\usepackage{amssymb}
\usepackage{pgfgantt}
\usepackage{hyperref}
\usepackage{dirtree}
\usepackage{xcolor,colortbl}
\usepackage{multirow}

\usepackage{bbding} %palomitas checkmark
\usepackage{pifont}

% A package which allows simple repetition counts, and some useful commands

\usepackage{forloop}
\newcounter{loopcntr}
\newcommand{\rpt}[2][1]{%
  \forloop{loopcntr}{0}{\value{loopcntr}<#1}{#2}%
}
\newcommand{\on}[1][1]{
  \forloop{loopcntr}{0}{\value{loopcntr}<#1}{&\cellcolor{gray}}
}
\newcommand{\off}[1][1]{
  \forloop{loopcntr}{0}{\value{loopcntr}<#1}{&}
}

\addtolength{\textheight}{90pt}

\newcommand{\I}{\mathbb{I}}
\newcommand{\K}{\mathbb{K}}
\newcommand{\N}{\mathbb{N}}
\newcommand{\Q}{\mathbb{Q}}
\newcommand{\R}{\mathbb{R}}
\newcommand{\Z}{\mathbb{Z}}

\newcommand{\specialcell}[2][c]{%
  \begin{tabular}[#1]{@{}c@{}}#2\end{tabular}}

\begin{document}
\maketitle
\begin{abstract}

  %La importancia de la rob\'{o}tica de servicios en la actualidad es innegable. Estos avances est\'{a}n revolucionando la forma en que interactuamos con el mundo, ofreciendo un amplio abanico de aplicaciones en diversos sectores. Desde veh\'{i}culos aut\'{o}nomos, robots m\'{o}viles en l\'{o}gistica, hasta la exploraci\'{o}n espacial, la rob\'{o}tica de servicios ha demostrado ser \'{u}til en entornos donde los seres humanos pueden enfrentar riesgos o dificultades.\\
  
  %Dentro de la rob\'{o}tica m\'{o}vil podemos encontrar robots a\'{e}reos, mejor conocidos como Veh\'{i}culos A\'{e}reos No Tripulados (VANT), estos robots m\'{o}viles tienen la capacidad de volar y acceder a lugares de manera r\'{a}pida y eficiente, convirti\'{e}ndolos en herramientas extremadamente vers\'{a}tiles. Veh\'{i}culos as\'{i} est\'{a}n siendo utilizados por empresas de comercio electr\'{o}nico para entregar productos a los clientes de manera \'{a}gil ((hablar del mTSP)), en la agricultura para monitorear cultivos e identificar problemas con plagas para aplicar pesticidas o fertilizantes de manera precisa. En el \'{a}mbito de la seguridad, pueden utilizarse para la vigilancia de \'{A}reas de dif\'{i}cil acceso o en situaciones de emergencia, proporcionando informaci\'{o}n valiosa en tiempo real a los equipos de rescate.\\

  %A pesar de los numerosos avances de la rob\'{o}tica de servicios, existen desaf\'{i}os y problem\'{a}ticas asociadas a la planificaci\'{o}n de trayectorias en ambientes desconocidos y cambiantes.\\
  %La \textbf{exploraci\'{o}n} implica una serie de tareas y desaf\'{i}os. Estos pueden incluir la \textbf{planificaci\'{o}n de rutas} para cubrir eficientemente el \'{a}rea a explorar, la \textbf{detecci\'{o}n} y \textbf{evaci\'{o}n de obst\'{a}culos}, \textbf{la localizaci\'{o}n y mapeo simult\'{a}neo} (SLAM) y la toma de decisiones para maximizar la informaci\'{o}n.\\

  %La \textbf{coordinaci\'{o}n} y el \textbf{trabajo en equipo} de m\'{u}ltiples-VANT(s) representa un desaf\'{i}o emocionante. La \textbf{colaboraci\'{o}n} de varios VANT(s) puede ser \'{u}til en misiones de b\'{u}squeda y rescate, en donde pueden cubrir \'{a}reas m\'{a}s extensas y realizar tareas m\'{a}s complejas de manera simult\'{a}nea. La coordinaci\'{o}n entre los VANT(s) puede optimizar la eficiencia de las operaciones y aumentar las posibilidades de \'{e}xito.\\

  %El objetivo de este trabajo es la propuesta de una arquitectura de software tolerante a fallas, capaz de explorar ambientes desconocidos y cambiantes para la coordinaci\'{o}n de Veh\'{i}culos A\'{e}reos No Tripulados.\\
  %El proyecto de investigaci\'{o}n demostrar\'{a} que es posible dise\~{n}ar algoritmos inteligentes de poca memoria capaces de resolver tareas en colaboraci\'{o}n multi-VANT.
  En las \'{u}ltimas decadas se ha visto una explosi\'{o}n en el inter\'{e}s de los Veh\'{i}culos A\'{e}reos No Tripulados (conocidos como VANT o coloquialmente drones), a la par que se han introducido nuevas tecnolog\'{i}as de comunicaci\'{o}n, percepci\'{o}n del ambiente (sensores) y computo. Los avances en comunicaci\'{o}n han sido aplicados al control de VANT, logrando crear soluciones en vigilancia, busqueda y rescate, soluciones para el problema de la \'{u}ltima milla, inclusive en espect\'{a}culos a\'{e}reos (solo por mencionar algunos). Dichas aplicaciones suelen carecer de autonom\'{i}a. Para que un robot se considere aut\'{o}nomo deber\'{a} tomar decisiones y realizar tareas sin necesidad de que alguien le diga qu\'{e} hacer o guiarlo paso a paso. Tener la capacidad de percibir su entorno utilizando diversos sensores y usar la informaci�n para decidir c�mo moverse. Lograr altos niveles de aut\'{o}nomia el robot debe resolver primero problemas como la localizaci\'{o}n, mapeo y navegaci\'{o}n.\\
  Se ha demostrado que es posible dotar de autonom\'{i}a a un robot (m\'{o}vil o a\'{e}reo) y las aplicaciones antes mencionadas son prueba de ello. Dichas soluciones parten de la ayuda de tener el problema de localizaci\'{o} resuelto al ser aplicaciones en exteriores y poder hacer uso de sensores como GPS. Los Veh\'{i}culos A\'{e}reos no Tripulados (VANT) del ma�ana deber\'{a}n de navegar en \'{a}reas urbanas de la mejor manera posible y tener la habilidad de trabajar en coordinaci\'{o}n multi-VANT.\\
  El enfoque de este trabajo es la creaci\'{o}n y propuesta de una arquitectura capaz de coordinar m\'{u}ltiples Veh\'{i}culos A\'{e}reos No Tripulados con algoritmos de baja memoria para la exploraci\'{o}n de \'{a}reas desconocidas y cambiantes convirtiendose en un desaf\'{i}o en rob\'{o}tica m\'{o}vil que busca coordinar y optimizar el movimiento de varios robots para explorar eficientemente un \'{a}rea desconocida. El objetivo pudiera ser maximizar la cobertura del \'{a}rea y minimizar el tiempo requerido para completar la exploraci\'{o}n. Este problema implica tomar decisiones complejas, como asignar tareas a los robots, evitar colisiones y planificar rutas \'{o}ptimas. Factores como la comunicaci\'{o}n entre robots, la incertidumbre del entorno y las limitaciones de recursos deben ser considerados.\\
  Resolver eficazmente este problema permitir\'{i}a mejorar la eficiencia y efectividad de las misiones de exploraci\'{o}n en diversos campos, como la b\'{u}squeda y rescate, la inspecci\'{o}n de infraestructuras y entre otras. \\
  \medskip \\
  
  \noindent \textbf{Palabras claves:} multi-VANT, coordinaci\'{o}n multi-VANT, exploraci\'{o}n 3D, planificador de rutas 3D.
  
\end{abstract}

\newpage
\section*{Datos Generales}

\subsection*{T\'{\i}tulo de proyecto}
Estrategias para la exploraci\'{o}n coordinada multi-VANT
\subsection*{Datos del alumno}
\begin{tabular}{ll} 
  Nombre:  &          Luis Alberto Ballado Aradias \\
  Matr\'{i}cula: &   220229860003\\
Direcci\'{o}n:   & Juan Jos\'{e} de La Garza \#909\\
                 & Colonia: Guadalupe Mainero C.P. 87130\\
Tel\'{e}fono (casa):    & +52 (833) 2126651\\
Tel\'{e}fono (lugar de trabajo):    & +52 (834) 107 0220 + Ext  \\
Direcci\'{o}n electr\'{o}nica: & luis.ballado@cinvestav.mx \\
URL: & https://luis.madlab.mx
\end{tabular}
\subsection*{Instituci\'{o}n}
\begin{tabular}{ll} 
Nombre:  &          CINVESTAV-IPN \\
Departamento:    &  Unidad Tamaulipas\\
Direcci\'{o}n:   &  Km 5.5 carretera Cd. Victoria - Soto la Marina.\\
                 &  Parque Cient\'{i}fico y Tecnol\'{o}gico TECNOTAM,\\
                 &  Ciudad Victoria, Tamaulipas, C.P. 87130\\
Tel\'{e}fono:    & (+52) (834) 107 0220\\
\end{tabular}
\subsection*{Beca de tesis}
\begin{tabular}{ll} 
Instituci\'{o}n otorgante:  &  CONAHCYT  \\
Tipo de beca:      & Maestr\'ia Nacional\\
Vigencia:    &   Septiembre 2022 - Agosto 2024
\end{tabular}

\subsection*{Datos del asesor}
\begin{tabular}{ll} 
Nombre:  &   Dr. Jos\'{e} Gabriel Ram\'{i}rez Torres \\
Direcci\'{o}n:   &   Km. 5.5 carretera Cd. Victoria - Soto la Marina\\
                 &  Parque Cient\'{i}fico y Tecnol\'{o}gico TECNOTAM\\
                 &  Ciudad Victoria, Tamaulipas, C.P. 87130\\
Tel\'{e}fono (oficina):    &  (+52) (834) 107 0220 Ext. 1014 \\ 
Instituci\'{o}n:    &  CINVESTAV-IPN \\ 
Departamento adscripci\'{o}n: &  Unidad Tamaulipas\\
Grado acad\'{e}mico: & Doctorado \\\\
Nombre:  &   Dr. Eduardo Arturo Rodr\'{i}guez Tello \\
Direcci\'{o}n:   &   Km. 5.5 carretera Cd. Victoria - Soto la Marina\\
                 &  Parque Cient\'{i}fico y Tecnol\'{o}gico TECNOTAM\\
                 &  Ciudad Victoria, Tamaulipas, C.P. 87130\\
Tel\'{e}fono (oficina):    &  (+52) (834) 107 0220 Ext. 1100\\ 
Instituci\'{o}n:    &  CINVESTAV-IPN \\ 
Departamento adscripci\'{o}n: &  Unidad Tamaulipas\\
Grado acad\'{e}mico: & Doctorado 
\end{tabular}

\newpage
\section*{Descripci\'{o}n del proyecto}

El proyecto de estrategias para la exploraci\'{o}n coordinada multi-VANT se centra en las ventajas de tener m\'{u}ltiples-VANT(s) trabajando en conjunto para mejorar la eficiencia y cobertura de la exploraci\'{o}n proponiendo una arquitectura de software descentralizada que permita la coordinaci\'{o}n eficiente de multi-VANT para tareas de exploraci\'{o}n en entornos desconocidos y cambiantes.

\subsection*{Antecedentes y motivaci\'{o}n para el proyecto}

\begin{itemize}
\item Hablar de la robotica de servicios
\item robotica movil
\item clasificar los diversos robots moviles aereos
\item los problemas en la robotica movil
\item slam
\item robots coolaborativos
\end{itemize}

Millones de Veh\'{i}culos A\'{e}reos No Tripulados, o tambi\'{e}n conocidos como drones, han presentado una adopci\'{o}n masiva en diferentes aplicaciones, desde usos civiles (b\'{u}squeda y rescate, monitoreo industrial, vigilancia), hasta aplicaciones militares [1]. La popularidad de los VANT(s) es atribuida a su movilidad.\\

La idea de utilizar m\'{u}ltiples robots a\'{e}reos en un sistema coordinado se basa en el comportamiento de los enjambres de animales, como las abejas o los p\'{a}jaros, que trabajan juntos de manera colaborativa para lograr objetivos comunes. Esta inspiraci\'{o}n biol\'{o}gica ha llevado al desarrollo de algoritmos y t\'{e}cnicas para coordinar y controlar m\'{u}ltiples VANT(s) en diferentes aplicaciones.\\

El inter\'{e}s en la investigaci\'{o}n e inovaci\'{o}n de soluciones con Veh\'{i}culos A\'{e}reos No Tripulados ha crecido exponencialmente en a\~{n}os recientes [2,7,8,9,10].\\

En recientes a\~{n}os, dotar a los VANT de inteligencia para explotar la informaci\'{o}n recolectada de sensores a bordo, ha sido y es un \'{a}rea estudiada en rob\'{o}tica m\'{o}vil \'{a}rea (construcci\'{o}n de mapas)[3]. Buscando probar diferentes teor\'{i}as de control, convirti\'{e}ndo los problemas t\'{i}picos de control 2D (p\'{e}ndulo inverso fijo) a un ambiente 3D, teniendo m\'{a}s variables a controlar para mantener el equilibrio del p\'{e}ndulo y al mismo tiempo lograr el movimiento y las maniobras deseadas del dron en el espacio tridimensional[4].\\

El despliegue r\'{a}pido de robots en situaciones de riesgo, b\'{u}squeda y rescate ha sido un \'{a}rea ampliamente estudiada en la rob\'{o}tica m\'{o}vil. Donde se han aplicado teor\'{i}as de grafos para la obtenci\'{o}n de la mejor ruta. Los comportamientos reactivos son primordiales si pensamos en un agente aut\'{o}nomo. Esa percepci\'{o}n que podemos tener los seres humanos para reaccionar a ciertos retos. Buscar la manera de crear una arquitectura tolerante a fallas, capaz de coordinar m\'{u}ltiples v\'{e}hiculos a\'{e}reos no trupulados a medida que incrementa o disminuye la oferta de VANT(s) disponibles.\\

La coordinaci\'{o}n de m\'{u}ltiples-VANT(s) ofrece numerosos beneficios y oportunidades en diversos campos y aplicaciones.

\begin{itemize}
\item Eficiencia y cobertura
\item Redundancia y tolerancia a fallos
\item Adaptabilidad a entornos din\'{a}micos
\item Distribuci\'{o}n de carga de trabajo
\item Aprendizaje colaborativo
\end{itemize}

\newpage
\section*{Planteamiento del problema}

La coordinaci\'{o}n de m\'{u}ltiples-VANT (Veh\'{i}culos A\'{e}reos No Tripulados) es un desaf\'{i}o complejo en el campo de la rob\'{o}tica y la exploraci\'{o}n de \'{a}reas desconocidas. A medida que la tecnolog\'{i}a de los Veh\'{i}culos A\'{e}reos No Tripulados contin\'{u}a avanzando y se vuelven m\'{a}s accesibles, se presenta la oportunidad de utilizar equipos de m\'{u}ltiples VANT(s) para realizar tareas de manera colaborativa y eficiente. Sin embargo, esta coordinaci\'{o}n planea diversas problem\'{a}ticas que deben abordarse.\\

La coordinaci\'{o}n de m\'{u}ltiples VANT(s) implica la necesidad de establecer una comunicaci\'{o}n efectiva entre ellos. Los VANT(s) deben intercambiar informaci\'{o}n relevante sobre su posici\'{o}n, estado, objetivos y otros datos importantes. La comunicaci\'{o}n debe ser confiable, de baja latencia y capaz de manejar m\'{u}ltiples enlaces de manera simult\'{a}nea. Adem\'{a}s, los protocolos de comunicaci\'{o}n deben ser seguros para proteger la integridad y confidencialidad de los datos transmitidos.\\

Otro desaf\'{i}o es la planificaci\'{o}n de rutas y la toma de decisiones distribuida. Los VANT(s) deben coordinar sus movimientos para evitar colisiones y lograr una cobertura eficiente del \'{a}rea objetivo. Esto implica la necesidad de desarrollar algoritmos y estrategias que permitan la planificaci\'{o}n de rutas din\'{a}micas, considerando los obst\'{a}culos y las restricciones del entorno. Adem\'{a}s, los VANT(s) deben tomar decisiones colaborativas para adaptarse a situaciones imprevistas o cambios en el entorno.\\

La asignaci\'{o}n de tareas tambi\'{e}n es un aspecto cr\'{i}tico en la coordinaci\'{o}n de m\'{u}ltiples VANT(s). Cada VANT puede tener diferentes capacidades y sensores especializados, por lo que es importante asignar tareas de acuerdo con las fortalezas individuales de cada robot. Adem\'{a}s, los VANT(s) deben colaborar en la recolecci\'{o}n y procesamiento de datos, evitanto la duplicaci\'{o}n de esfuerzos optimizando el uso de los recursos disponibles.\\

Dada un \'{a}rea de inter\'{e}s $A$ desconocida que se desea explorar,
\begin{itemize}
\item Un conjunto de Veh\'{i}culos A\'{e}reos No Tripulados (VANT) denotados como $V = V_{1},V_{2},V_{3},...,V_{n}$, donde $n$ es el n\'{u}mero total de VANT's disponibles
\item Un conjunto de tareas de exploraci\'{o}n denotados como $T = T_{1}, T_{2}, T_{3}, T_{m}$, donde $m$ es el n\'{u}mero total de tareas a realizar.
\end{itemize}

restricciones y requisitos espec\'{i}ficos del problema, como l\'{i}mites de tiempo, obst\'{a}culos a evitar, etc.

Para cada tarea de exploraci\'{o}n $T_{m}$, se definen las siguientes variables:

\begin{itemize}
\item Posici\'{o}n inicial: $p_{i}(x,y,z)$, representa la posici\'{o}n inicial del VANT o los m\'{u}ltiples-VANTs asignados a la tarea $T_{m}$
\item Trayectoria: $\alpha_{i}$, describe la trayectoria seguida por el/los VANT(s) asignado(s) a la tarea $T_{m}$ en funci\'{o}n del tiempo $t$
\item Informaci\'{o}n recolectada: $C_{i}$, representa la informaci\'{o}n recolectada por el/los VANT(s) asignado(s) durante la exploraci\'{o}n
\end{itemize}

La funci\'{o}n objetivo variar\'{a} seg\'{u}n los objetivos espec\'{i}ficos del problema.
\begin{itemize}
\item Maximizar la cobertura del \'{a}rea de inter\'{e}s $A$
\item Minimizar el tiempo total requerido para cubrir el \'{a}rea de inter\'{e}s $A$
\item Maximizar la cantidad de informaci\'{o}n recolectada
\end{itemize}

\newpage
\section*{Objetivos generales y espec\'{\i}ficos del proyecto}

\textbf{General} \\

Dise�ar una arquitectura de software descentralizada de baja memoria capaz de resolver los problemas de localizaci\'{o}n, mapeo, navegaci\'{o}n y coordinaci\'{o}n multi-VANT en ambientes desconocidos y din\'{a}micos para tareas de exploraci\'{o}n en interiores.

%Desarrollo e implementaci\'{o}n de una arquitectura de software tolerante a fallas para la coordinaci\'{o}n de m\'{u}ltiples VANT(s) aplicados a una simulaci\'{o}n de b\'{u}squeda y rescate.

%El objetivo general de la tesis es desarrollar estrategias efectivas para la coordinaci�n de m\'{u}ltiples Veh\'{i}culos A\'{e}reos no Tripulados, con el fin de mejorar la eficiencia y el rendimiento en aplicaciones en exploraciones a�reas.

\bigskip
\noindent
%\textbf{Particulares} \\
De manera m\'{a}s espec\'{i}fica, se encuentr\'{a}n los siguientes objetivos: %espec\'{i}ficos:

\begin{enumerate}

  %
%\item Garantizar que los VANT(s) eviten colisiones entre ellos y con objetos en su entorno.
%\item Eficiencia y rendimiento del sistema en su conjunto. Asignar tareas de manera \'{o}ptima entre los m\'{u}ltiples-VANT(s), minimizando los tiempos de espera y de respuesta.
%\item Garantizar que cada VANT contribuya de manera efectiva al logro de los objetivos generales, sin redundancia ni superposici\'{o}n de tareas.
%\item Comunicaci\'{o}n efectiva entre los m\'{u}ltiples-VANT(s) para intercambiar informaci\'{o}n y sincronizar sus acciones. El objetivo es establecer una comunicaci\'{o}n confiable y eficiente que permita la transmisi\'{o}n de datos relevantes y las instrucciones necesarias para la coordinaci\'{o}n.
%\item Adaptabilidad y flexibilidad: Los objetivos de la coordinaci\'{o}n pueden cambiar en funci\'{o}n de las circunstancias y las necesidades. La coordinaci\'{o}n de m\'{u}ltiples-VANT(s) debe ser adaptable y flexible para ajustarse a cambios en el entorno, nuevos objetivos o la incorporaci\'{o}n o salida de VANT(s) del sistema.
\item Evaluar las limitantes de las soluciones en la literatura asociados con la coordinaci\'{o}n multi-VANT. Enfoc\'{a}ndose en aspectos como la comunicaci\'{o}n, evasi\'{o}n de obst\'{a}culos, asignaci\'{o}n de tareas y sincronizaci\'{o}n de informaci\'{o}n para la creaci\'{o}n de una propuesta.
%, as\'{i} como conocer las soluciones propuestas en la literatura.
\item Determinar una herramienta de simulaci\'{o}n de libre uso para rob\'{o}tica compatible con el middleware ROS (Robot Operating System), capaz de importar modelos tridimencionales del ambiente y del robot para lograr el control de m\'{u}ltiples agentes mediante la ejecuci\'{o}n de diversos algoritmos en lenguajes de programaci\'{o}n standard en la rob\'{o}tica pudiendo ejecutar y simular los programas de la arquitectura de coordinaci\'{o}n propuesta.
  
%\item Generaci\'{o}n y manipulaci\'{o}n del modelo din\'{a}mico de un VANT en el simulador elegido.
%\item Desarrollo e implementaci\'{o}n de tres t\'{e}cnicas para la representaci\'{o}n del ambiente en 3D, as\'{i} como el desarrollo de tres planificadores de trayectorias para un \'{u}nico VANT.
%\item Desarrollo de una estrategia para la coordinaci\'{o}n multi-VANT en base al mejor candidato del punto anterior.
%\item Realizar simulaciones y pruebas de la estrategia propuesta, evaluando su desempe\~{n}o en t\'{e}rminos de tiempo de respuesta, eficiencia en la asignaci\'{o}n de recursos, seguridad y capacidad de adaptaci\'{o}n a situaciones din\'{a}micas.
\item Comparar y analizar los resultados obtenidos con enfoques existentes en la coordinaci\'{o}n multi-VANT, demostrando las ventajas y desventajas de la estrategia propuesta. Buscando proponer recomendaciones y pautas pr\'{a}cticas para la implementaci\'{o}n y aplicaci\'{o}n de la estrategias de coordinaci\'{o}n multi-VANT en escenarios reales, considerando factores como la escalabilidad, la robustez y los recursos computacionales requeridos.
%\item Escribir un art\'{i}culo cient\'{i}fico con los hallazgos en esta tesis en relaci\'{o}n con la exploraci\'{o}n y coordinaci\'{o}n multi-VANT.
\end{enumerate}

\section*{Metodolog\'{\i}a}

La metodolog\'{i}a propuesta se divide en tres etapas, iniciando en septiembre del 2023. A continuaci\'{o}n se detallan cada una de las actividades que se plantean realizar en cada una.

\subsection*{Etapa 1. An\'{a}lisis y dise\~{n}o de la soluci\'{o}n propuesta}

En esta etapa se comprende en la revisi\'{o}n de la literatura de manera m\'{a}s completa, que permita contar con la informaci\'{o}n necesaria para la elecci\'{o}n de los mejores algoritmos para planificaci\'{o}n de trayectorias. Una vez realizada la elecci\'{o}n del algoritmo que se usar\'{a} para la propuesta de arquitectura, se proceder\'{a} a revisar y estudiar las arquitecturas para los robots colaborativos. Finalmente, se realizar\'{a} el dise\~{n}o de la arquitectura.\\
Las actividades espec\'{i}ficas a realizarse en la etapa 1, son:
  
  \begin{enumerate}
  \item \textbf{E1.A1.} Realizar una revisi\'{o}n de la literatura sobre coordinaci\'{o}n multi-VANT.
  \item \textbf{E1.A2.} Revisar y documentar
  \item \textbf{E1.A3.} Estudiar los problemas
  \item \textbf{E1.A4.} Seleccionar 
  \item \textbf{E1.A5.} Definir la arquitectura
  \item \textbf{E1.A6.} Proponer el dise\~{n}o
  \item \textbf{E1.A7.} Elaborar la documentaci\'{o}n de la revisi\'{o}n bibliogr\'{a}fica realizada, que formar\'{a} parte de la tesis
  \item \textbf{E1.A8.} Realizar la documentaci\'{o}n del trabajo realizado en la etapa
  \end{enumerate}
  
  \subsection*{Etapa 2. Implementaci\'{o}n y validaci\'{o}n}
  
  Esta etapa se centra en la implementaci\'{o}n del dise\~{n}o del sistema ...\\
  Las actividades espec\'{i}ficas a realizarse en la etapa 2, son:

    \begin{enumerate}
    \item \textbf{E2.A1.} Definir
    \item \textbf{E2.A2.} Definir
    \item \textbf{E2.A3.} Realizar
    \item \textbf{E2.A4.} Implementar
    \item \textbf{E2.A5.} Documentar
    \item \textbf{E2.A6.} Elaborar
    \end{enumerate}


    \subsection*{Etapa 3. Evaluaci\'{o}n experimental, resultados y conclusi\'{o}n}
    
    Lorem itsum....\\
    Las actividades espec\'{i}ficas a realizarse en la etapa 3, son:
      
    \begin{enumerate}
    \item \textbf{E3.A1.} Experimentos para evaluar el prototipo creado en la etapa anterior en las m\'{e}tricas de inter\'{e}s
    \item \textbf{E3.A2.} Recabar
    \item \textbf{E3.A3.} Escribir un art\'{i}culo cient\'{i}fico, con los hallazgos de esta tesis
    \item \textbf{E3.A4.} Iniciar el proceso de revisi\'{o}n de tesis con los directores.
    \item \textbf{E3.A5.} Escribir los cap\'{i}tulos correspondientes a la implementaci\'{o}n, resultados y conclusiones
    \item \textbf{E3.A6.} Iniciar el proceso de graduaci\'{o}n
    \item \textbf{E3.A7.} Participar en actividades de difusi\'{o}n y divulgaci\'{o}n
    \end{enumerate}

    
    \begin{enumerate}
\item Revisi\'{o}n de literatura
  \begin{itemize}
  \item Realizar una revisi\'{o}n de la literatura cientifica y t\'{e}cnica relacionada con la coordinaci\'{o}n de m\'{u}ltiples VANT(s).
  \item Identificar los enfoques existentes, algoritmos y tecnolog\'{i}as utilizadas en la coordinaci\'{o}n de m\'{u}ltiples VANT(s).
  \end{itemize}
\item An\'{a}lisis y dise\~{n}o de la soluci\'{o}n propuesta
  \begin{itemize}
  \item Identificar los requisitos clave para una coordinaci\'{o}n eficiente de VANT(s), considerando factores como la seguiridad, la eficiencia energ\'{e}tica y la capacidad de adaptaci\'{o}n a diferentes entornos.
  \item Establecer m\'{e}tricas y criterios de evaluaci\'{o}n para medir el desempe\~{n}o de la coordinaci\'{o}n de m\'{u}ltiples VANTS.
  \item Dise\~{n}ar algoritmos y protocolos de comunicaci\'{o}n que permitan la coordinaci\'{o}n de manera eficiente.
    \item Proponer estrategias para la asignaci\'{o}n de tareas y la gesti\'{o}n de recursos en funci\'{o}n de los requisitos identificados.
  \end{itemize}
\item Implementaci\'{o}n y validaci\'{o}n
  \begin{itemize}
  \item Implementar la metodolog\'{i}a propuesta utilizando lenguajes de programaci\'{o}n adecuados y herramientas de simulaci\'{o}n.
  \item Realizar simulaciones para evaluar el desempe\~{n}o de la coordinaci\'{o}n de m\'{u}ltiples VANT(s) bajo diferentes escenarios.
  \end{itemize}
\item Evaluaci\'{o}n, resultados y conclusiones

  \begin{itemize}
  \item Analizar y comparar los resultados obtenidos con otros enfoques existentes.
  \item Extraer conclusiones sobre la efectividad propuesta en t\'{e}rminos de coordinaci\'{o}n eficiente de Veh\'{i}culos A\'{e}reos No Tripulados.
  \item Identificar posibles mejoras y \'{a}reas de investigaci\'{o}n futuras en el campo de la coordinaci\'{o}n de m\'{u}ltiples VANT(s).
  \end{itemize}
\end{enumerate}

\newpage
\section*{Cronograma de actividades (plan de trabajo)}

\noindent\begin{tabular}{p{0.27\textwidth}*{12}{|p{0.04\textwidth}}|}
% The top line
\textbf{Cuatrimestre}
& \multicolumn{4}{c|}{Q1} 
& \multicolumn{4}{c|}{Q2} 
& \multicolumn{4}{c|}{Q3}\\ 
           
% The second line, with its five years of four quarters
\rpt[3]{& 1 & 2 & 3 & 4} \\
\hline
\textcolor{red}{Actividad 1}\\
% using the on macro to fill in twenty cells as `on'
%\specialcell{Actividad 1\\espacio}        \on[0] \off[12] \\
%Actividad 1    \on[0] \off[12]\\
\hline
\textbf{E1.A1}    \on[2] \off[10] \\
\hline
\textbf{E1.A2}    \on[2]  \off[10] \\
\hline
% using the on macro followed by the off macro
\textcolor{red}{Actividad 2}\\
\hline
\textbf{E2.A1}    \on[2] \off[10] \\
\hline
\textbf{E2.A2}    \on[2]  \off[10] \\
\hline
Actividad 3\\
\hline
\textbf{E3.A1}    \on[2] \off[6] \on[2] \off[2] \\
\hline
\textbf{E3.A2}    \off[2] \on[4] \off[4] \on[1] \off[1] \\
\hline
% Note the omitting the count to on or off is the same as setting the count to 1
Actividad 7    \off[11] \on \\
\hline
\end{tabular}

\iffalse
\begin{ganttchart}[vgrid={draw=none, dotted}]{1}{12}
\gantttitlelist{1,...,12}{1} \\
\ganttbar{}{1}{4} \\
\ganttbar{}{5}{11}
\end{ganttchart}
\fi

\iffalse
\definecolor{barblue}{RGB}{153,204,254}
\definecolor{groupblue}{RGB}{51,102,254}
\definecolor{linkred}{RGB}{165,0,33}
\renewcommand\sfdefault{phv}
\renewcommand\mddefault{mc}
\renewcommand\bfdefault{bc}
\setganttlinklabel{s-s}{START-TO-START}
\setganttlinklabel{f-s}{FINISH-TO-START}
\setganttlinklabel{f-f}{FINISH-TO-FINISH}
\sffamily
\begin{ganttchart}[
    canvas/.append style={fill=none, draw=black!5, line width=.75pt},
    hgrid style/.style={draw=black!5, line width=.75pt},
    vgrid={*1{draw=black!5, line width=.75pt}},
    today=7,
    today rule/.style={
      draw=black!64,
      dash pattern=on 3.5pt off 4.5pt,
      line width=1.5pt
    },
    today label font=\small\bfseries,
    title/.style={draw=none, fill=none},
    title label font=\bfseries\footnotesize,
    title label node/.append style={below=7pt},
    include title in canvas=false,
    bar label font=\mdseries\small\color{black!70},
    bar label node/.append style={left=2cm},
    bar/.append style={draw=none, fill=black!63},
    bar incomplete/.append style={fill=barblue},
    bar progress label font=\mdseries\footnotesize\color{black!70},
    group incomplete/.append style={fill=groupblue},
    group left shift=0,
    group right shift=0,
    group height=.5,
    group peaks tip position=0,
    group label node/.append style={left=.6cm},
    group progress label font=\bfseries\small,
    link/.style={-latex, line width=1.5pt, linkred},
    link label font=\scriptsize\bfseries,
    link label node/.append style={below left=-2pt and 0pt}
  ]{1}{13}
  \gantttitle[
    title label node/.append style={below left=7pt and -3pt}
  ]{CUATRIMESTRE:\quad1}{1}
  \gantttitlelist{2,...,13}{1} \\
  \ganttgroup[progress=57]{WBS 1 Summary Element 1}{1}{10} \\
  \ganttbar[
    progress=75,
    name=WBS1A
  ]{\textbf{WBS 1.1} Activity A}{1}{8} \\
  \ganttbar[
    progress=67,
    name=WBS1B
  ]{\textbf{WBS 1.2} Activity B}{1}{3} \\
  \ganttbar[
    progress=50,
    name=WBS1C
  ]{\textbf{WBS 1.3} Activity C}{4}{10} \\
  \ganttbar[
    progress=0,
    name=WBS1D
  ]{\textbf{WBS 1.4} Activity D}{4}{10} \\[grid]
  \ganttgroup[progress=0]{WBS 2 Summary Element 2}{4}{10} \\
  \ganttbar[progress=0]{\textbf{WBS 2.1} Activity E}{4}{5} \\
  \ganttbar[progress=0]{\textbf{WBS 2.2} Activity F}{6}{8} \\
  \ganttbar[progress=0]{\textbf{WBS 2.3} Activity G}{9}{10}
  \ganttlink[link type=s-s]{WBS1A}{WBS1B}
  \ganttlink[link type=f-s]{WBS1B}{WBS1C}
  \ganttlink[
    link type=f-f,
    link label node/.append style=left
  ]{WBS1C}{WBS1D}
  \end{ganttchart}
\fi

\section*{Infraestructura}

Para el desarrollo de este proyecto de investigaci\'{o}n, se har\'{a} uso de un equipo de c\'{o}mputo con las siguientes caracter\'{i}sticas:

\begin{itemize}
\item iMac (21.5-inch, Late 2015)
\item Procesador 2.8 GHz Quad-Core Intel Core i5
\item Memoria Ram 8 GB 1867 MHz DDR3
\item Graphics Intel Iris Pro Graphics 6200 1536 MB
\item Almacenamiento 1 TB
\item Tarjeta Raspberry Pi para Nodos ROS
\end{itemize}

\newpage
\section*{Estado del arte}

\textbf{mencionar solo trabajos en percepcion(lidar,c�mara), colaboracion de robots moviles y aereos, como representan su entorno y llenar la tabla de comparaciones}

\dirtree{%
  .1 Rob\'{o}tica M\'{o}vil.
  .2 Problemas en rob\'{o}tica m\'{o}vil.
  .3 Mapas.
  .3 Localizaci\'{o}n.
  .3 Planificaci\'{o}n trayectorias.
  .2 Rob\'{o}tica M\'{o}vil A\'{e}rea.
  .3 Din\'{a}mica de un Veh\'{i}culo A\'{e}reo No Tripulado.
  .3 Control de un Veh\'{i}culo A\'{e}reo No Tripulado.
  .2 Construcci\'{o}n y representaci\'{o}n de mapas 3D.
  .3 Percepci\'{o}n.
  .4 Sensores LIDAR.
  .4 Odometr\'{i}a Visual.
  .2 Rob\'{o}tica Colaborativa.
  .3 Exploraci\'{o}n con m\'{u}ltiples VANT(s).
  .3 Coordinaci\'{o}n.
  .3 Colaboraci\'{o}n.
  .3 Arquitectura de software en rob\'{o}tica colaborativa.
}

\vspace{1cm}

Las aplicaciones de la rob\'{o}tica se han centrado en realizar tareas simples y repetitivas. La necesidad de robots con capacidad de identificar cambios en su entorno y reaccionar sin la intervenci\'{o}n humana, da origen a los robots inteligentes. Aunado a ello si deseamos que el robot se mueva libremente, los cambios en su entorno pueden aumentar r\'{a}pidamente y complicar el problema de un comportamiento inteligente. Dentro de la rob\'{o}tica m\'{o}vil inteligente se han propuesto estrategias de comportamiento reactivas, algoritmos que imitan el comportamiento de insectos y el c\'{o}mo se desplanzan en un entorno.\\
El objetivo principal de los algoritmos de navegaci\'{o}n es el de guiar al robot desde el punto de inicio al punto destino. Los trabajos por V. Lumelsky y A. Stephanov, et al. [11], dieron respuesta a problematicas de navegaci\'{o}n eficiente y de poca memoria (Algoritmos tipo bug).\\
Se considera a P. Hart, N. Nilsson et al. como los creadores del algoritmo A* en 1968 [12], al mejorar el algoritmo de Dijkstra para el robot Shakey, que deb\'{i}a navegar en una habitaci\'{o}n que conten\'{i}a obst\'{a}culos fijos. El objetivo principal del algoritmo A* es la eficiencia en la planificaci\'{o}n de rutas.\\
Otros algoritmos propuestos por A. Stentzz[13] han demostrado operar de manera eficiente ante obst\'{a}culos din\'{a}micos, a comparaci\'{o}n del algoritmo A* que vuelve a ejecutarse al encontrarse con un obst\'{a}culo, el algoritmo D* usa la informaci\'{o}n previa para buscar una ruta hacia el objetivo.\\

%La planificación de trayectorias también ha abordado la problemática de la planificación de múltiples robots. Se han desarrollado algoritmos que permiten a los robots colaborar y coordinarse para evitar colisiones y mejorar la eficiencia en sus tareas. Estos enfoques utilizan técnicas de planificación centralizada o descentralizada, y pueden basarse en métodos de búsqueda o algoritmos de optimización multiobjetivo.\\

La colaboraci\'{o}n de m\'{u}ltiples VANTs (veh\'{i}culos a\'{e}reos no tripulados), tambi\'{e}n conocidos como VANTs, ha surgido como una \'{a}rea de investigaci\'{o}n prometedora en los \'{u}ltimos a\~{n}os [1,2,3,5]. La capacidad de coordinar y colaborar entre s\'{i} permite a los VANTs realizar tareas complejas de manera eficiente, abriendo nuevas posibilidades en una amplia gama de aplicaciones, desde la vigilancia y la log\'{i}stica hasta la exploraci\'{o}n y la respuesta a desastres [1,2].\\

Uno de los desaf\'{i}os clave en la colaboraci\'{o}n de m\'{u}ltiples VANTs es la planificaci\'{o}n de rutas. Se han desarrollado diversos algoritmos para optimizar la planificaci\'{o}n de rutas dentro de la rob\'{o}tica m\'{o}vil, minimizando la colisi\'{o}n y mejorando la eficiencia de sus misiones[5,6]. Estos algoritmos tienen en cuenta varios factores, como las restricciones de vuelo, la energ\'{i}a restante de los VANTs y las ubicaciones objetivo, para generar trayectorias seguras y eficientes.\\

Adem\'{a}s de la planificaci\'{o}n de rutas, la coordinaci\'{o}n de los VANTs requiere una comunicaci\'{o}n efectiva. Se han investigado diferentes protocolos de comunicaci\'{o}n y estrategias de intercambio de informaci\'{o}n para permitir la colaboraci\'{o}n entre los VANTs. Algunos enfoques utilizan comunicaci\'{o}n directa entre los VANTs, mientras que otros emplean una arquitectura de red donde los VANTs se comunican a trav\'{e}s de una infraestructura centralizada[6]. La elecci\'{o}n del enfoque depende de las caracter\'{i}sticas de la aplicaci\'{o}n y las restricciones del sistema.\\

%La asignación de tareas es otro aspecto crucial en la colaboración de múltiples VANTs. Los VANTs deben ser capaces de dividir y asignar las tareas de manera óptima, considerando factores como la capacidad de carga, la distancia a las ubicaciones objetivo y los recursos disponibles. Se han propuesto diferentes estrategias de asignación de tareas, como algoritmos basados en la teoría de grafos y enfoques basados en técnicas de optimización.\\

La colaboraci\'{o}n de m\'{u}ltiples VANTs tambi\'{e}n puede implicar la formaci\'{o}n de formaciones o la realizaci\'{o}n de tareas coordinadas. Para ello, se han desarrollado algoritmos de control distribuido que permiten a los VANTs mantener posiciones relativas estables y realizar movimientos coordinados. Estos algoritmos[14] pueden basarse en t\'{e}cnicas de seguimiento y control de formaciones, y se han aplicado en diferentes contextos, desde la inspecci\'{o}n de infraestructuras hasta la b\'{u}squeda y rescate.\\

En t\'{e}rminos de validaci\'{o}n y evaluaci\'{o}n, se utilizan simulaciones y pruebas reales para verificar el rendimiento y la eficacia de los sistemas de colaboraci\'{o}n de m\'{u}ltiples VANTs. Las simulaciones permiten evaluar diferentes escenarios y ajustar los par\'{a}metros del sistema antes de las pruebas reales. Los casos de prueba reales proporcionan informaci\'{o}n sobre la implementaci\'{o}n y la eficiencia en situaciones del mundo real, y pueden ayudar a identificar desaf\'{i}os adicionales que deben abordarse.\\

%La planificación de trayectorias en robótica móvil es un campo de investigación fundamental que se enfoca en desarrollar algoritmos y técnicas para que los robots móviles puedan determinar rutas óptimas y seguras para navegar en entornos complejos. Esta área ha experimentado avances significativos en las últimas décadas, impulsada por el creciente interés en aplicaciones como la navegación autónoma, la logística y la robótica de servicio. A continuación, se presenta un estado del arte sobre la planificación de trayectorias en robótica móvil.\\

%Un enfoque común en la planificación de trayectorias es la búsqueda basada en grafos. Los algoritmos de búsqueda en grafos, como el algoritmo A*, permiten encontrar rutas óptimas en entornos discretizados. Estos algoritmos generan un grafo que representa el espacio de configuración del robot, donde los nodos son posiciones posibles y las aristas representan transiciones entre ellas. Sin embargo, estos enfoques enfrentan desafíos en entornos de alta dimensionalidad y con obstáculos dinámicos, ya que la construcción y búsqueda del grafo pueden volverse computacionalmente costosas.\\


La adquisici\'{o}n de datos es el primer paso en la representaci\'{o}n de mapas 3D con VANTs. Los VANTs pueden llevar a cabo vuelos sobre un \'{a}rea de inter\'{e}s, capturando im\'{a}genes desde diferentes \'{a}ngulos y alturas[15]. Estas t\'{e}cnicas aprovechan la informaci\'{o}n de correspondencia entre las im\'{a}genes para calcular la posici\'{o}n y orientaci\'{o}n relativa de las c\'{a}maras y reconstruir la estructura tridimensional del entorno.\\

Los VANTs pueden utilizar sensores LiDAR (Light Detection and Ranging) para capturar datos 3D. Los sensores LiDAR emiten pulsos de luz l\'{a}ser y miden el tiempo que tarda en reflejarse en los objetos circundantes. Esto permite obtener informaci\'{o}n precisa sobre la distancia y la posici\'{o}n tridimensional de los objetos en el entorno. Los datos LiDAR pueden combinarse con las im\'{a}genes capturadas para generar mapas 3D completos y detallados.\\

\begin{tabular}{ |p{1cm}||p{1cm}|p{2cm}|p{1.5cm}|p{2cm}|p{1.7cm}|p{1.7cm}|p{2.5cm}|  }
 \hline
 \multicolumn{8}{|c|}{Trabajos relacionados} \\
 \hline
 Ref.&3D&Localizaci\'{o}n&Mapa&Navegaci\'{o}n&Simulador&Percepci\'{o}n&Arquitectura\\
 \hline
 \cite{Smith2020}&\ding{51}&BLE&Voronoi&A*&gazebo&LiDAR&Centralizada\\
 \cite{Smith2020}&\ding{55}&GPS&Celdas&D* Lite&gazebo&Camara&Descentralizada\\
 \cite{Smith2020}&\ding{51}&Feat.&Grafo&RRT*&matlab&Camara&Centralizada\\
 \cite{Smith2020}&\ding{55}&GPS&Oct-tree&A*&webots&LiDAR&Descentralizada\\
 \cite{Smith2020}&\ding{55}&BLE&Grafo&RRT&airsim&siete&Descentralizada\\
 
 \hline
\end{tabular}

%La visualización y la interacción con los mapas 3D también han sido objeto de investigación. Se han desarrollado herramientas de visualización interactiva que permiten a los usuarios explorar y analizar los mapas 3D generados. Estas herramientas pueden incluir capacidades de navegación, análisis de datos y anotación de objetos para facilitar la comprensión y el uso de los mapas 3D en aplicaciones específicas.\\

\newpage
\section*{Contribuciones o resultados esperados}

\begin{enumerate}
\item Documentaci\'{o}n, y c\'{o}digos liberados
  \begin{itemize}
  \item Algoritmo para la planificaci\'{o}n de rutas
  \item Protocolos de comunicaci\'{o}n y coordinaci\'{o}n
  \item Coordinaci\'{o}n en entornos din\'{a}micos
  \end{itemize}
\item Simulaci\'{o}n de soluci\'{o}n
  \begin{itemize}
  \item Simulaciones detalladas y pruebas en entornos controlados
  \item M\'{e}tricas como tiempo de respuesta, consumo de energ\'{i}a y la capacidad de adaptaci\'{o}n a diferentes escenarios. 
  \end{itemize}
\item Tesis impresa. \cite{Einstein1905}  \cite{Smith2020}
\item Divulgar este trabajo en algun espacio afin en ingenieria y tecnologia.
\end{enumerate}

\newpage
%\section*{Referencias}
\bibliographystyle{ieeetr}
\bibliography{referencias.bib}
%\begin{enumerate}
%\item  H. Shakhatreh et al., 'Unmanned Aerial Vehicles: A Survey on Civil Applications and Key Research Challenges', arXiv:1805.00881, 2018
%\item P. Daponte et al., 'Metrology for drone and drone for metrology: Measurement systems on small civilian drones', in Metrology for Aerospace (MetroAeroSpace), 2015 IEEE, 2015, pp. 306-311: IEEE.
%\item A. Shukla and H. Karki, 'Application of robotics in onshore oil and gas industry A review Part I', Robotics and Autonomous Systems, vol. 75, pp. 490-507, 2016
%\item M. Hehn and R. D'Andrea, 'A flying inverted pendulum', 2011 IEEE International Conference on Robotics and Automation, Shanghai, China, 2011, pp. 763-770, doi: 10.1109/ICRA.2011.5980244.
%\item Z. Fu, Y. Mao, D. He, J. Yu and G. Xie, 'Secure Multi-UAV Collaborative Task Allocation,' in IEEE Access, vol. 7, pp. 35579-35587, 2019, doi: 10.1109/ACCESS.2019.2902221.
%\item B. Zhou, H. Xu and S. Shen, 'RACER: Rapid Collaborative Exploration With a Decentralized Multi-UAV System,' in IEEE Transactions on Robotics, vol. 39, no. 3, pp. 1816-1835, June 2023, doi: 10.1109/TRO.2023.3236945.
%\item 'Hovering over the drone patent landscape, ifi claims patent services, Nov 2014 \href{https://www.ificlaims.com/news/view/blog-posts/hovering-over-the-drone.htm}{online}
%\item L. Gupta, R. Jain, and G. Vaszkun, 'Survey of important issues in UAV communication networks', IEEE Communications Surveys \& Tutorials, vol. 18, no. 2, pp. 1123-1152, 2016.
%\item J. Senthilnath, M. Kandukuri, A. Dokania, and K. Ramesh, 'Application of UAV imaging platform for vegetation analysis based on spectral-spatial methods', Computers and Electronics in Agriculture, vol. 140, pp. 8-24, 2017.
%\item H. Zhou, H. Kong, L. Wei, D. Creighton, and S. Nahavandi, 'On detecting road regions in a single UAV image,' IEEE Trans. Intell. Transp. Syst., vol. 18, no. 7, pp. 1713-1722, 2017.
%\item V. Lumelsky y A. Stephanov, Path-Planning Strategies for a Point Mobile Automaton Moving Amidst Unknown Obstacles of Arbitrary Shapes, Algorithmica, vol. 2, pp. 403-430, 1987.
%\item Peter E. Hart, Nils J. Nilsson, Bertram Raphael, A Formal Basis for the Heuristic Determination of Minimum Cost Paths, IEEE Transactions on Systems Science and Cybernetics, vol. 4, pág 100-107, 1968
%\item A. Stentz, Optimal and efficient path planning for partially-known environments, Proc. of IEEE Conference on Robotic Automation, pág 1058-1068, 1994
%\item L. Barnes, W. Alvis, M. Fields, K. Valavanis, and W. Moreno, 'Swarm formation control with potential fields formed by bivariate normal functions,' in Control and Automation, 2006. MED'06. 14th Mediterranean Conference on, 2006, pp. 1-7: IEEE.
%\item T. Cieslewski, E. Kaufmann and D. Scaramuzza, "Rapid exploration with multi-rotors: A frontier selection method for high speed flight," 2017 IEEE/RSJ International Conference on Intelligent Robots and Systems (IROS), Vancouver, BC, Canada, 2017, pp. 2135-2142, doi: 10.1109/IROS.2017.8206030.
%\item Morbidi, F.; Cano, R.; Lara, D. Minimum-energy path generation for a quadrotor UAV. In Proceedings of the IEEE InternationalvConference on Robotics and Automation, Stockholm, Sweden, %16–21 May 2016; pp. 1492–1498.
%\item Zhang, X.Y.; Duan, H.B. An improved constrained differential evolution algorithm for unmanned aerial vehicle global route planning. Appl. Soft Comput. 2015, 26, %270–284
%\item Chen, Y.; Luo, G.; Mei, Y.; Yu, J.; Su, X. UAV Path Planning Using Artificial Potential Field Method Updated by Optimal Control Theory. Int. J. Syst. Sci. 2014, 47, %1407–1420
%\item Huang, S.; Teo, R.S.H. Computationally Efficient Visibility Graph-Based Generation Of 3D Shortest Collision-Free Path Among Polyhedral Obstacles For Unmanned Aerial Vehicles. In Proceedings of the International Conference on Unmanned Aircraft Systems, Atlanta, GA, USA, %11–14 June 2019; pp. 1218–1223.
%\item Maini, P.; Sujit, P.B. Path planning for a UAV with kinematic constraints in the presence of polygonal obstacles. In Proceedings of the International Conference on Unmanned Aircraft Systems, Arlington, VA, USA,% 7–10 June 2016; pp. %62–67.
%\item Canny, J.; Reif, J. New lower bound techniques for robot motion planning problems. In Proceedings of the 28th Annual Symposium on Foundations of Computer Science, Los Angeles, CA, USA, %12–14 October 1987; pp. 49–60.
%\end{enumerate}

\newpage
\begin{center}
  \begin{tabular}{c@{\hspace{5em}}c}
{\Large{Fecha de inicio}} & {\Large{Fecha de terminaci\'on}} \\
% Poner fechas respectivas
&\\
Septiembre de 2023 & Agosto de 2024
\end{tabular} \vspace{2.5cm} \\
Firma del alumno: \underline{\hspace{5cm}} \vspace{2cm}\\ \ \\
{\Large{Comit\'e de aprobaci\'on del tema de tesis}} \vspace{2cm} \\
\begin{tabular}{p{7cm}p{5cm}}
Dr. Jos\'{e} Gabriel Ram\'{i}rez Torres & \underline{\hspace{5cm}} \vspace{1cm} \\
Dr. Eduardo Arturo Rodr\'{i}guez Tello   & \underline{\hspace{5cm}} \vspace{1cm} \\
Dr. 3 & \underline{\hspace{5cm}} \vspace{1cm} \\
Dr. 4 & \underline{\hspace{5cm}} %\vspace{1cm} \\
\end{tabular}
\end{center}
%\newpage
%\bibliographystyle{plain}
%\bibliography{c:/RodRuiz/bib}
%\bibliography{book,jour,kocc,proc,trep}

\end{document}

