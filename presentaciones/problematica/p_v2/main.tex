%%%%%%%%%%%%%%%%%%%%%%%%%%%%%%%%%%%%%%%%%
% Lachaise Assignment
% LaTeX Template
% Version 1.0 (26/6/2018)
%
% This template originates from:
% http://www.LaTeXTemplates.com
%
% Authors:
% Marion Lachaise & François Févotte
% Vel (vel@LaTeXTemplates.com)
%
% License:
% CC BY-NC-SA 3.0 (http://creativecommons.org/licenses/by-nc-sa/3.0/)
% 
%%%%%%%%%%%%%%%%%%%%%%%%%%%%%%%%%%%%%%%%%

%----------------------------------------------------------------------------------------
%	PACKAGES AND OTHER DOCUMENT CONFIGURATIONS
%----------------------------------------------------------------------------------------

\documentclass{article}

\input{structure.tex}

\title{Estrategias para la exploración coordinada multi-VANT}

\author{Luis Ballado\\ \texttt{luis.ballado@cinvestav.mx}} % Author name and email address

\date{CINVESTAV UNIDAD TAMAULIPAS --- \today} 

\begin{document}

\maketitle

En la última década el uso de los Vehículos Aéreos No Tripulados (VANTs) en aplicaciones civiles han tenido mucho auge, trayendo consigo la necesidad de coordinarlos a medida que aumenta su uso en diversas aplicaciones como vigilancia, búsqueda y rescate. Donde la coordinación efectiva y segura de múltiples VANTs se vuelve fundamental. Dicha tarea de coordinar plantea desafíos para evitar colisiones, asignar tareas, planificar rutas, administrar recursos para optimizar el rendimiento de las misiones asignadas y garantizar una comunicación fluida son aspectos críticos. Se requiere el desarrollo de algoritmos adaptables, sistemas de control y comunicaciones robustas, así como regulaciones adecuadas para abordar esta problemática y permitir el despliegue eficiente de los VANTs en entornos complejos.\\

Resolver estas problemáticas permitirá aprovechar todo el potencial de los VANTs en diversas aplicaciones, como la vigilancia, entrega de paquetes, la inspección de infraestructuras, hasta la respuesta rápida a emergencias donde un VANT pueda acceder de forma eficiente.\\

\textbf{Palabras clave: } \textit{VANTs, coordinación multiagente, mapeo, inteligencia colectiva, percepción y autonomía}\\



\end{document}


