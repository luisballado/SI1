%%%%%%%%%%%%%%%%%%%%%%%%%%%%%%%%%%%%%%%%%
% Lachaise Assignment
% LaTeX Template
% Version 1.0 (26/6/2018)
%
% This template originates from:
% http://www.LaTeXTemplates.com
%
% Authors:
% Marion Lachaise & François Févotte
% Vel (vel@LaTeXTemplates.com)
%
% License:
% CC BY-NC-SA 3.0 (http://creativecommons.org/licenses/by-nc-sa/3.0/)
% 
%%%%%%%%%%%%%%%%%%%%%%%%%%%%%%%%%%%%%%%%%

%----------------------------------------------------------------------------------------
%	PACKAGES AND OTHER DOCUMENT CONFIGURATIONS
%----------------------------------------------------------------------------------------

\documentclass{article}

%%%%%%%%%%%%%%%%%%%%%%%%%%%%%%%%%%%%%%%%%
% Lachaise Assignment
% Structure Specification File
% Version 1.0 (26/6/2018)
%
% This template originates from:
% http://www.LaTeXTemplates.com
%
% Authors:
% Marion Lachaise & François Févotte
% Vel (vel@LaTeXTemplates.com)
%
% License:
% CC BY-NC-SA 3.0 (http://creativecommons.org/licenses/by-nc-sa/3.0/)
% 
%%%%%%%%%%%%%%%%%%%%%%%%%%%%%%%%%%%%%%%%%

%----------------------------------------------------------------------------------------
%	PACKAGES AND OTHER DOCUMENT CONFIGURATIONS
%----------------------------------------------------------------------------------------

\usepackage{amsmath,amsfonts,stmaryrd,amssymb} % Math packages

\usepackage{enumerate} % Custom item numbers for enumerations

\usepackage[ruled]{algorithm2e} % Algorithms

\usepackage[framemethod=tikz]{mdframed} % Allows defining custom boxed/framed environments
\usepackage[spanish]{babel}
\usepackage{hyperref}
\usepackage{listings} % File listings, with syntax highlighting
\lstset{
	basicstyle=\ttfamily, % Typeset listings in monospace font
}

%----------------------------------------------------------------------------------------
%	DOCUMENT MARGINS
%----------------------------------------------------------------------------------------

\usepackage{geometry} % Required for adjusting page dimensions and margins

\geometry{
	paper=a4paper, % Paper size, change to letterpaper for US letter size
	top=2.5cm, % Top margin
	bottom=3cm, % Bottom margin
	left=2.5cm, % Left margin
	right=2.5cm, % Right margin
	headheight=14pt, % Header height
	footskip=1.5cm, % Space from the bottom margin to the baseline of the footer
	headsep=1.2cm, % Space from the top margin to the baseline of the header
	%showframe, % Uncomment to show how the type block is set on the page
}

%----------------------------------------------------------------------------------------
%	FONTS
%----------------------------------------------------------------------------------------

\usepackage[utf8]{inputenc} % Required for inputting international characters
\usepackage[T1]{fontenc} % Output font encoding for international characters

\usepackage{XCharter} % Use the XCharter fonts

%----------------------------------------------------------------------------------------
%	COMMAND LINE ENVIRONMENT
%----------------------------------------------------------------------------------------

% Usage:
% \begin{commandline}
%	\begin{verbatim}
%		$ ls
%		
%		Applications	Desktop	...
%	\end{verbatim}
% \end{commandline}

\mdfdefinestyle{commandline}{
	leftmargin=10pt,
	rightmargin=10pt,
	innerleftmargin=15pt,
	middlelinecolor=black!50!white,
	middlelinewidth=2pt,
	frametitlerule=false,
	backgroundcolor=black!5!white,
	frametitle={Command Line},
	frametitlefont={\normalfont\sffamily\color{white}\hspace{-1em}},
	frametitlebackgroundcolor=black!50!white,
	nobreak,
}

% Define a custom environment for command-line snapshots
\newenvironment{commandline}{
	\medskip
	\begin{mdframed}[style=commandline]
}{
	\end{mdframed}
	\medskip
}

%----------------------------------------------------------------------------------------
%	FILE CONTENTS ENVIRONMENT
%----------------------------------------------------------------------------------------

% Usage:
% \begin{file}[optional filename, defaults to "File"]
%	File contents, for example, with a listings environment
% \end{file}

\mdfdefinestyle{file}{
	innertopmargin=1.6\baselineskip,
	innerbottommargin=0.8\baselineskip,
	topline=false, bottomline=false,
	leftline=false, rightline=false,
	leftmargin=2cm,
	rightmargin=2cm,
	singleextra={%
		\draw[fill=black!10!white](P)++(0,-1.2em)rectangle(P-|O);
		\node[anchor=north west]
		at(P-|O){\ttfamily\mdfilename};
		%
		\def\l{3em}
		\draw(O-|P)++(-\l,0)--++(\l,\l)--(P)--(P-|O)--(O)--cycle;
		\draw(O-|P)++(-\l,0)--++(0,\l)--++(\l,0);
	},
	nobreak,
}

% Define a custom environment for file contents
\newenvironment{file}[1][File]{ % Set the default filename to "File"
	\medskip
	\newcommand{\mdfilename}{#1}
	\begin{mdframed}[style=file]
}{
	\end{mdframed}
	\medskip
}

%----------------------------------------------------------------------------------------
%	NUMBERED QUESTIONS ENVIRONMENT
%----------------------------------------------------------------------------------------

% Usage:
% \begin{question}[optional title]
%	Question contents
% \end{question}

\mdfdefinestyle{question}{
	innertopmargin=1.2\baselineskip,
	innerbottommargin=0.8\baselineskip,
	roundcorner=5pt,
	nobreak,
	singleextra={%
		\draw(P-|O)node[xshift=1em,anchor=west,fill=white,draw,rounded corners=5pt]{%
		Question \theQuestion\questionTitle};
	},
}

\newcounter{Question} % Stores the current question number that gets iterated with each new question

% Define a custom environment for numbered questions
\newenvironment{question}[1][\unskip]{
	\bigskip
	\stepcounter{Question}
	\newcommand{\questionTitle}{~#1}
	\begin{mdframed}[style=question]
}{
	\end{mdframed}
	\medskip
}

%----------------------------------------------------------------------------------------
%	WARNING TEXT ENVIRONMENT
%----------------------------------------------------------------------------------------

% Usage:
% \begin{warn}[optional title, defaults to "Warning:"]
%	Contents
% \end{warn}

\mdfdefinestyle{warning}{
	topline=false, bottomline=false,
	leftline=false, rightline=false,
	nobreak,
	singleextra={%
		\draw(P-|O)++(-0.5em,0)node(tmp1){};
		\draw(P-|O)++(0.5em,0)node(tmp2){};
		\fill[black,rotate around={45:(P-|O)}](tmp1)rectangle(tmp2);
		\node at(P-|O){\color{white}\scriptsize\bf !};
		\draw[very thick](P-|O)++(0,-1em)--(O);%--(O-|P);
	}
}

% Define a custom environment for warning text
\newenvironment{warn}[1][Warning:]{ % Set the default warning to "Warning:"
	\medskip
	\begin{mdframed}[style=warning]
		\noindent{\textbf{#1}}
}{
	\end{mdframed}
}

%----------------------------------------------------------------------------------------
%	INFORMATION ENVIRONMENT
%----------------------------------------------------------------------------------------

% Usage:
% \begin{info}[optional title, defaults to "Info:"]
% 	contents
% 	\end{info}

\mdfdefinestyle{info}{%
	topline=false, bottomline=false,
	leftline=false, rightline=false,
	nobreak,
	singleextra={%
		\fill[black](P-|O)circle[radius=0.4em];
		\node at(P-|O){\color{white}\scriptsize\bf i};
		\draw[very thick](P-|O)++(0,-0.8em)--(O);%--(O-|P);
	}
}

% Define a custom environment for information
\newenvironment{info}[1][Info:]{ % Set the default title to "Info:"
	\medskip
	\begin{mdframed}[style=info]
		\noindent{\textbf{#1}}
}{
	\end{mdframed}
}


\title{Estrategias para la exploración coordinada multi-VANT}

\author{Luis Ballado\\ \texttt{luis.ballado@cinvestav.mx}} % Author name and email address

\date{CINVESTAV UNIDAD TAMAULIPAS --- \today} 

\begin{document}

\maketitle

En la última década el uso de los Vehículos Aéreos No Tripulados (VANTs) en aplicaciones civiles han tenido mucho auge, trayendo consigo la necesidad de coordinarlos a medida que aumenta su uso en diversas aplicaciones como vigilancia, búsqueda y rescate. Donde la coordinación efectiva y segura de múltiples VANTs se vuelve fundamental. Dicha tarea de coordinar plantea desafíos para evitar colisiones, asignar tareas, planificar rutas, administrar recursos para optimizar el rendimiento de las misiones asignadas y garantizar una comunicación fluida son aspectos críticos. Se requiere el desarrollo de algoritmos adaptables, sistemas de control y comunicaciones robustas, así como regulaciones adecuadas para abordar esta problemática y permitir el despliegue eficiente de los VANTs en entornos complejos.\\

Resolver estas problemáticas permitirá aprovechar todo el potencial de los VANTs en diversas aplicaciones, como la vigilancia, entrega de paquetes, la inspección de infraestructuras, hasta la respuesta rápida a emergencias donde un VANT pueda acceder de forma eficiente.\\

\textbf{Palabras clave: } \textit{VANTs, coordinación multiagente, mapeo, inteligencia colectiva, percepción y autonomía}\\

En la robótica móvil la construcción y/o uso de un modelo del ambiente (un mapa), es uno de los principales problemas en el área. Es casi imposible para un robot operar en un ambiente si un mapa que guíe sus movimientos.\\

Así como la construcción de mapas en áreas desconocidas presenta un desafío dentro de la róbotica móvil, dentro de la robótica móvil aérea también lo es. Teniendo problemas inherentes que deben abordarse para lograr un funcionamiento eficiente y seguro como lo son:

\begin{itemize}
\item Establidad y control: Los VANTs deben ser capaces de mantener una estabilidad adecuada y un control preciso durante el vuelo. Esto implica superar desafios como la estabilización en condiciones climáticas adversas y la capacidad de responder rápidamente a las perturbaciones externas.
\item Navegación y planificación de rutas - Los VANTs debe poder navegar de manera autónoma y planificar rutas óptimas para alcanzar sus objetivos. Esto implica la detección y evación de obstáculos, la planificación de trayectorias suaves y eficientes, y la capacidad de adaptarse a entornos desconocidos y cambiantes.
\item Detección y percepción - Los VANTs necesitan sistemas de detección y percepción para obtener información sobre sy entorno y tomar decisiones. Esto incluye la capacidad de detectar y reconocer objetos, evitar colisiones, realizar mapeo y localización simultánea (SLAM).
\item Administración de energía - La administración eficiente es un desafío, especialmente en robots que operan de manera autónoma durante largos períodos. Maximizar la duración de la batería y optimizar el consumo de energía son aspectos clave en el diseño y desarrollo de robots móviles.
\end{itemize}

A medida que la tecnología avanza, surgen nuevos desafíos y áreas de investigación para mejorar aún más el rendimiento y capacidad de los robots móviles aéreos.\\

La generación de algoritmos adaptativos capaces de ajustar su comportamiento en función de las nuevas condiciones del entorno es la pieza clave que se busca en este trabajo.\\

El uso y aplicaciones de vehículos aéreos han sido extensamente estudiados y una larga lista de aplicaciones  ya existen entre nosotros. Pero las limitaciones que pueden presentar en la aplicación de un solo agente pueden reducirse a medida que aumentamos el número de agentes en multiples VANTs.\\

Una de las aplicaciones con mayor éxito de multiples VANTs se encuentra en el área del entretenimiento. Siendo la compañia Genesis (la marca de lujo de Hyundai) con una coreografía de \href{https://www.guinnessworldrecords.com/news/commercial/2021/5/3281-drones-break-dazzling-record-for-most-airborne-simultaneously-655062}{3,281} drones perfectamente sincronizados.\\

%como circula la informacion
%si es centralizado
%si tiene una base centralizada
%sobre el que se quiere hacer
%que desafios
%para que lo quiero hacer
%logistica que son para un unico vants pero un numeroso numero de vants puede tener una gran ventaja en ciertas tareas
%en lugar de tener un dron especializado, se puede
%para que .. que pretendo tener con ellos .. obtener robustes de que alguno pueda fallar ...
%desarrollar estrategias para realizar una exploracion eficiente con tolerancia a fallos
%como lo quiero lo puedo hacer .. que sea desentralizado ..distribuido entre los drones

%que el utilizar los multiples drones
%una de las areas es la exploracion ahi se puede mencionar exploracion busqueda y rescate .. vigilancia..inspeccion en campos eolicos..todas se reducen a generar un mapa para tomar decisiones de que se va a reaizar

%como -- los efoques .. pocos drones..enjambre es mas de 20 drones.. la aplicacion de grupos de robots.. enjambres

%leer tesis de juan carlos

Además de su exitoso empleo en espectáculos, los VANTs han tomado un gran interés en diferentes tipos de industrias donde el despliege coordinado de vehículos áereos es importante como en: Seguridad y vigilancia, Transportación colaborativa, Monitoreo de ambientes, búsqueda y rescate; sólo por mencionar algunos.\\

Un grupo de robots áereos puede presentar un comportamiento de enjambre mediante la integración de mecanismos de coordinación en su control.\\

Los metodos de coordinación pueden ser pensados como herramientas de propósito general en la planeación de objetivos a nivel de enjambre. Las tareas de enjambre pueden ser el control de formación, evación de obstáculos y optimización en areas de exploración.\\

Para la aplicación de algortimos de coordinación, cada robot áereo debe poseer sensores que lo ayuden a percibir el ambiente en el que se desplaza, así como conocer el estado de las variables de sus vecinos como posición, velocidad y altitud. Aunque el intercambio de información puede resultar un alto costo computacional a medida que el enjambre de robots áeros crece.\\

\textbf{Planteamiento formal del problema de exploración utilizando múltiples VANTs:}\\

Objetivo: Realizar la exploración de manera eficiente, cubriendo la mayor cantidad posible del área de interés $A$ con los drones disponibles.

\begin{itemize}
\item Dado un área de interés $A$ que se desea explorar
\item Un conjunto de drones denotados como $D={D_1,D_2,...,D_n}$, donde n es el número total de drones disponibles.
\item Un conjunto de tareas de exploración o mapeo, denotado como $T={T_1,T_2,...,T_m}$, donde m es el número total de tareas a realizar.
\item Restricciones y requisitos específicos del problema, como límites de tiempo, áreas prioritarias, obstáculos a evitar, etc.
\end{itemize}

Para cada tarea de exploración $Tm$, se definen las siguientes variables:

\begin{itemize}
\item Posición inicial: $p_i$ (x,y,z), representa la posición inicial del dron o los drones asignados a la tarea $Tm$
\item Trayectoria: $\alpha_i$, describe la trayectoria seguida por el/los dron/es asignado(s) a la tarea $Tm$ en función del tiempo t.
\item Información recolectada: $C_i$, representa la información recolectada por el/los dron/nes asignado(s) a la tarea $Tm$ durante la exploración.
\end{itemize}

La función objetivo puede variar según los objetivos específicos del problema. Algunas posibles funciones objetivo podrían ser:

\begin{itemize}
\item Maximizar la cobertura del área de interés A.
\item Minimizar el tiempo total requerido para cubrir el área de interés A.
\item Maximizar la cantidad de información recolectada.
\end{itemize}

\end{document}


